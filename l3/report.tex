\documentclass{article}

\usepackage{polski}
\usepackage{amsmath}
\usepackage{amsfonts}
\usepackage{algorithm2e}

\SetKwInput{KwData}{Dane}
\SetKwInput{KwResult}{Wynik}

\author{Paweł Wilkosz}
\title{Obliczenia naukowe - sprawozdanie 3.}

\begin{document}
\maketitle

\section{Opis problemu}

Celem zadania jest znalezienie miejsc zerowych funkcji $f: \mathbb{R} \to \mathbb{R}$ przy wykorzystaniu algorytmów iteracyjnych.
Dla każdego algorytmu definiujemy dwa parametry precyzji.
\begin{itemize}
  \item $\epsilon$ - precyzja wartości funkcji
  \item $\delta$ - precyzja argumentów funkcji
\end{itemize}

\section{Algorytmy}

\subsection{Metoda bisekcji}
Metoda ta wymaga podania przedziału $[a,b]$, na którym dana funkcja $f$ zmienia znak.
Iteracja metody polega na dzieleniu przedziału na dwie części i wybrania do kolejnej iteracji tej, na której funkcja zmienia znak.
Algorytm kończy działanie gdy długość przedziału jest mniejsza niż dane $\delta$ lub wartość funkcji dla początku przedziału jest mniejsza niż $\epsilon$.

\begin{algorithm}[H]
  \KwData{f, a, b, $\delta$, $\epsilon$}
  \KwResult{r, f(r), it, error}

  u $\leftarrow$ f(a)
  v $\leftarrow$ f(b)
  e $\leftarrow$ b-a

  \If{sgn(u) == sgn(v)}{
    return (0, 0, 0, 1)
  }

  it $\leftarrow$ 1

  \While{true}{
    e $\leftarrow$ e/2;
    c $\leftarrow$ a+e;
    w $\leftarrow$ f(c);

    \If{$abs(e) < \delta$ or $|w| < \epsilon$}{
      return (c, w, it, 0)
    }

    \eIf{sgn(w) != sgn(u)}{
      b $\leftarrow$ c
      v $\leftarrow$ w
    }{
      a $\leftarrow$ c
      u $\leftarrow$ w
    }

    it++
  }
\end{algorithm}


\subsection{Metoda Newtona}

W tej metodzie potrzebna jest nam pochodna funkcji oraz wartość początkowa $x_0$.
W każdej kolejnej iteracji algorytmu przy pomocy danej pochodnej wyznaczamy styczną do funkcji w punkcie $x_0$.
Następnie badamy wartość funkcji w punkcie przecięcia z osią X - $x_1$
Jeżeli wartość funkcji w tym punkcie nie jest odpowiednio bliska zeru podstawiamy $x_0 = x_1$ i kontunuujemy algorytm.
W przeciwnym wypadku zwracamy znalezioną wartość.

Algorytm działa błędnie w dwóch przypadkach: gdy wartość pochodna będzie bliska zeru lub jeżeli nie znajdziemy odpowiedniej wartości w \texttt{maxit} iteracjach.

\begin{algorithm}[H]
  \KwData{f, $f'$ x0, $\delta$, $\epsilon$, maxit}
  \KwResult{r, f(r), it, error}

  v $\leftarrow$ f(x0)

  \If{$abs(v) < \epsilon$}{
    return (x0, v, 0, 0)
  }

  \For{it $\leftarrow$ 1 \KwTo maxit}{
    \If{$abs(f'(x0)) < \epsilon$}{
      return (0, 0, it, 2)
    }
    x1 $\leftarrow$ x0 - v/$f'$(x0);
    v $\leftarrow$ f(x1)

    \If{$abs(x1-x0) < \delta$ or $abs(v) < epsilon$}{
      return (x1, v, it, 0)
    }

    x0 $\leftarrow$ x1
  }

  return (0, 0, 0, 1)

\end{algorithm}

\subsection{Metoda siecznych}

Ta metoda polega na wyznaczeniu ciągu punktów zbieżnego do miejsca zerowego definiowanego za pomocą wzoru
$$
x_{n+2} = \text{punkt przecięcia z osią OX siecznej przecinającej funkcję w punktach } x_n \text{ i } x_{n+1}
$$

Program kończy prace gdy odległość pomiędzy kolejnymi punktami ciągu bedzie mniejsza niż $\delta$ lub wartość funkcji bedzie mniejsza niż $\epsilon$.

Algorytm może skończyć się błędem gdy odpowiednia wartość nie zostanie znaleziona w \texttt{maxit} iteracjach.

\begin{algorithm}[H]
  \KwData{f, x0, x1, $\delta$, $\epsilon$, maxit}
  \KwResult{r, f(r), it, error}

  a $\leftarrow$ x0;
  b $\leftarrow$ x1;
  fa $\leftarrow$ f(a);
  fb $\leftarrow$ f(b);

  \For{it $\leftarrow$ 1 \KwTo maxit}{
    \If{$abs(fa) > abs(fb)$}{
      a $\leftrightarrow$ b;
      fa $\leftrightarrow$ fb
    }

    s $\leftarrow$ (b-a) / (fb-fa)

    b $\leftarrow$ a;
    fb $\leftarrow$ fa

    a $\leftarrow$ a - fa*s
    
    fa $\leftarrow$ f(a)

    \If{$abs(b-a) < \delta$ or $abs(fa) < epsilon$}{
      return (a, fa, it, 0)
    }
  }

  return (0, 0, 0, 1)
  

\end{algorithm}

\section{Przykłady}

\section{Wnioski}



\end{document}