\documentclass{article}

\usepackage{polski}
\usepackage{amsmath}
\usepackage{amsfonts}
\usepackage{algorithm2e}

\SetKwInput{KwData}{Dane}
\SetKwInput{KwResult}{Wynik}

\author{Paweł Wilkosz}
\title{Obliczenia naukowe - sprawozdanie 3.}

\begin{document}
\maketitle

\section{Opis problemu}

Celem zadania jest znalezienie miejsc zerowych funkcji $f: \mathbb{R} \to \mathbb{R}$ przy wykorzystaniu algorytmów iteracyjnych.
Dla każdego algorytmu definiujemy dwa parametry precyzji.
\begin{itemize}
  \item $\epsilon$ - precyzja wartości funkcji
  \item $\delta$ - precyzja argumentów funkcji
\end{itemize}

\section{Algorytmy}

\subsection{Metoda bisekcji}
Metoda ta wymaga podania przedziału $[a,b]$, na którym dana funkcja $f$ zmienia znak.
Iteracja metody polega na dzieleniu przedziału na dwie części i wybrania do kolejnej iteracji tej, na której funkcja zmienia znak.
Algorytm kończy działanie gdy długość przedziału jest mniejsza niż dane $\delta$ lub wartość funkcji dla początku przedziału jest mniejsza niż $\epsilon$.

\begin{algorithm}[H]
  \KwData{f, a, b, $\delta$, $\epsilon$}
  \KwResult{r, f(r), it, error}

  u $\leftarrow$ f(a)
  v $\leftarrow$ f(b)
  e $\leftarrow$ b-a

  \If{sgn(u) == sgn(v)}{
    return (0, 0, 0, 1)
  }

  it $\leftarrow$ 1

  \While{true}{
    e $\leftarrow$ e/2;
    c $\leftarrow$ a+e;
    w $\leftarrow$ f(c);

    \If{$abs(e) < \delta$ or $|w| < \epsilon$}{
      return (c, w, it, 0)
    }

    \eIf{sgn(w) != sgn(u)}{
      b $\leftarrow$ c
      v $\leftarrow$ w
    }{
      a $\leftarrow$ c
      u $\leftarrow$ w
    }

    it++
  }
\end{algorithm}


\subsection{Metoda Newtona}

W tej metodzie potrzebna jest nam pochodna funkcji oraz wartość początkowa $x_0$.
W każdej kolejnej iteracji algorytmu przy pomocy danej pochodnej wyznaczamy styczną do funkcji w punkcie $x_0$.
Następnie badamy wartość funkcji w punkcie przecięcia z osią X - $x_1$
Jeżeli wartość funkcji w tym punkcie nie jest odpowiednio bliska zeru podstawiamy $x_0 = x_1$ i kontunuujemy algorytm.
W przeciwnym wypadku zwracamy znalezioną wartość.

Algorytm działa błędnie w dwóch przypadkach: gdy wartość pochodna będzie bliska zeru lub jeżeli nie znajdziemy odpowiedniej wartości w \texttt{maxit} iteracjach.

\begin{algorithm}[H]
  \KwData{f, $f'$ x0, $\delta$, $\epsilon$, maxit}
  \KwResult{r, f(r), it, error}

  v $\leftarrow$ f(x0)

  \If{$abs(v) < \epsilon$}{
    return (x0, v, 0, 0)
  }

  \For{it $\leftarrow$ 1 \KwTo maxit}{
    \If{$abs(f'(x0)) < \epsilon$}{
      return (0, 0, it, 2)
    }
    x1 $\leftarrow$ x0 - v/$f'$(x0);
    v $\leftarrow$ f(x1)

    \If{$abs(x1-x0) < \delta$ or $abs(v) < epsilon$}{
      return (x1, v, it, 0)
    }

    x0 $\leftarrow$ x1
  }

  return (0, 0, 0, 1)

\end{algorithm}

\subsection{Metoda siecznych}

Ta metoda polega na wyznaczeniu ciągu punktów zbieżnego do miejsca zerowego definiowanego za pomocą wzoru
$$
x_{n+2} = \text{punkt przecięcia z osią OX siecznej przecinającej funkcję w punktach } x_n \text{ i } x_{n+1}
$$

Program kończy prace gdy odległość pomiędzy kolejnymi punktami ciągu bedzie mniejsza niż $\delta$ lub wartość funkcji bedzie mniejsza niż $\epsilon$.

Algorytm może skończyć się błędem gdy odpowiednia wartość nie zostanie znaleziona w \texttt{maxit} iteracjach.

\begin{algorithm}[H]
  \KwData{f, x0, x1, $\delta$, $\epsilon$, maxit}
  \KwResult{r, f(r), it, error}

  a $\leftarrow$ x0;
  b $\leftarrow$ x1;
  fa $\leftarrow$ f(a);
  fb $\leftarrow$ f(b);

  \For{it $\leftarrow$ 1 \KwTo maxit}{
    \If{$abs(fa) > abs(fb)$}{
      a $\leftrightarrow$ b;
      fa $\leftrightarrow$ fb
    }

    s $\leftarrow$ (b-a) / (fb-fa)

    b $\leftarrow$ a;
    fb $\leftarrow$ fa

    a $\leftarrow$ a - fa*s
    
    fa $\leftarrow$ f(a)

    \If{$abs(b-a) < \delta$ or $abs(fa) < epsilon$}{
      return (a, fa, it, 0)
    }
  }

  return (0, 0, 0, 1)
  

\end{algorithm}

\section{Przykłady}

\subsection{Zadanie 4}

Celem zadania było przetestowanie powyższych metod dla równania
$\sin x - (\frac{1}{2}x)^2 = 0$

Dane wejściowe dla poszczególnych metod to
\begin{itemize}
  \item metoda bisekcji - przedział początkowy $[1.5, 2.0]$,
  \item metoda Newtona - przybliżenie początkowe $x_0 = 1.5$,  
  \item metoda siecznych - przybliżenia początkowe $x_0 = 1.0, x_1 = 2.0$
\end{itemize}
Dla wszystkich metod zastosowano parametry dokładności $\delta = \frac{1}{2}10^{-5}$ oraz $\epsilon = \frac{1}{2}10^{-5}$.

Wyniki eksperymentu są przedstawione w poniższej tabeli.

\begin{tabular}{|c|c|c|c|}
\hline
metoda & $x$ & $f(x)$ & liczba iteracji \\
\hline
bisekcji &  1.9337539672851562   &    -2.7027680138402843e-7 & 16 \\
Newtona & 1.933753779789742  & -2.2423316314856834e-8 & 4  \\
siecznych & 1.933753644474301 & 1.564525129449379e-7 & 4 \\
\hline
\end{tabular}

Można zaobserwować, że metoda Newtona osiągnęła najdokładniejszy wynik, a metoda bisekcji wymagała znacznie większej liczby iteracji niż pozostałe algorytmy.

\subsection{Zadanie 5}

Celem zadania było użycie metody bisekcji do znalezienie punktu przecięcia funkcji $f(x) = e^x$ i $g(x) = 3x$ przy dokładności $\delta = 10^{-4}$ i $\epsilon = 10^{-4}$.

W celu dobrania odpowiednich przedziałów nalezy dokonać wstępnej analizy problemu.

Łatwo zauważyć, że dla $x < 0$ funkcje te nie przecinają się, ponieważ $g$ na osiąga wartości ujemne podczas gdy $f$ jest zawsze dodatnie.
Możemy też zaobserwować, że $f(0) = 1 > 0 = g(0)$ i $f(1) = e < 3 = g(1)$
Zatem jako pierwszy przeszukiwany przedział wybieramy $[0,1]$.
Z racji na wyższe tempo wzrostu funkcji wykładniczej, funkcje powinny się przeciąć kolejny raz dla $x>1$.
W związku z tym przeszukujemy również przedział $[1,10]$.

Wyniki eksperymentów przedstawione są w poniższej tabeli.

\begin{tabular}{|c|c|c|c|}
\hline
przedział & $x$ & $f(x)$ & liczba iteracji \\
\hline
$[0,1]$ & 0.619140625 & 9.066320343276146e-5 & 9 \\
$[1,10]$ & 1.5121002197265625 & 5.274503124397256e-5 & 16 \\
\hline
\end{tabular}

Jak widać, wstępna analiza była poprawna i w podanych przedziałach udało się znaleźć miejsca zerowe.
Rozległość drugiego przedziału zwiększyła jednak potrzebną liczbę iteracji.

\subsection{Zadanie 6}

Zadanie polegało na użyciu wszystkich zaimplementowanych metod do zbadania miejsc zerowych funkcji
$
f_1(x) = e^{1-x} - 1
$
i
$
f_2(x) = xe^{-x}
$
przy dokładności $\delta = 10^{-5}$ i $\epsilon = 10^{-5}$.

W celu poprawnego zadania należy dokonać wstępnej analizy danych funkcji.
Podstawienie przykładowych wartości do $f_1$ pozwala stwierdzić że jest to funkcja malejąca oraz dodatnia w zerze
Na tej podstawie, dla metody bisekcji użyjemy przedziału $[0,10]$, dla metody Newtona za punkt startowy wybierzemy $0$, a dla metody siecznych jako wartości startowych użyjemy $0$ i $0.1$.
Wyniki powyższego eksperymentu widoczne są w poniższej tabeli.

\begin{tabular}{|c|c|c|c|}
\hline
metoda    & $x$                & $f(x)$                & liczba iteracji \\ \hline
bisekcji  & 1.0000038146972656 & -3.814689989667386e-6 & 19              \\ \hline
Newtona   & 0.9999984358892101 & 1.5641120130194253e-6 & 4               \\ \hline
siecznych & 0.9999999935791432 & 6.420856957234378e-9  & 6               \\ \hline
\end{tabular}

Jak widać, wszystkie metody zwróciły wynik bliski poprawnemu ($1$), jednak najlepszy wynik osiągnęła metoda siecznych.

Wstępna analiza dla $f_2$ pozwala znaleźć trywialny pierwiastek równy $0$.
Dla metody bisekcji wybieramy zatem przedział $[-1,1]$, dla metody Newtona wybieramy wartość startową $-1$, a jako wartości początkowe dla metody siecznych obieramy $-1$ i $-0.5$.
Wyniki tych testów widoczne są w poniższej tabeli.

\begin{tabular}{|c|c|c|c|}
\hline
metoda    & $x$                    & $f(x)$                 & liczba iteracji \\ \hline
bisekcji  & 0.0                    & 0,0                    & 1               \\ \hline
Newtona   & -3.0642493416461764e-7 & -3.0642502806087233e-7 & 5               \\ \hline
siecznych & -3.304447675535891e-8  & -3.304447784729637e-8  & 7               \\ \hline
\end{tabular}

Jak widać dobranie przedziału, symetrycznego wokół miejsca zerowego mocno faworyzuje metodę bisekcji.
Po raz kolejny metoda siecznych okazuje się dokładniejsza od metody Newtona.

Kolejny eksperyment polega na uźyciu $x_0>1$ w metodzie Newtona dla obu funkcji oraz $x_0=1$ dla $f_2$.
Jako przykładowej wartości $x_0>1$ użyjemy liczby $50$.

Uruchomienie funkcji z powyższymi parametrami powoduje błąd zbyt małej pochodnej dla $x_0>1$ dla $f_1$ i $x_0=1$ dla $f_2$.
Jest to spowodowane niskim tempem wzrostu funkcji dla tych wartości.
Użycie duzego $x_0$ w metodzie Newtona dla $f_2$ powoduje zwrócenie fałszywego miejsca zerowego w wartości początkowej.
Jest to spowodowane granicą funkcji $f_2$ w nieskończoności równą $0$.

\section{Wnioski}

Alorytmy numerycznego wyznaczania miejsc zerowych funkcji są trudne do sklasyfikowania.
Ich dokładność oraz poprawność działania zależy od danego przypadku i wstępnej analizy problemu.

Metoda bisekcji zdaje się być najbardziej niezawodna, jednak wymaga znalezienia odpowiedniego przedziału oraz kończy pracę po największej liczbie iteracji.
Metoda siecznych zwraca najdokładniejsze wyniki, jednak wybranie odpowiednich wartości startowych wymaga odpowiedniej analizy.
Również w przypadku metody Newtona wyniki są zależne od wartości poczatkowej i istnieją dane, dla których metoda może zwracać błędne wyniki.

Powyższe algorytmy wykazują wysoką skuteczność w powierzonych zadaniach jednak bez odpowiedniej analizy istnieje duże ryzyko błędu.

Poszczególne metody narażone są na różne błędy, zatem w celu wyeliminowania ryzyka w krytycznych przypadkach, poza przeanalizowaniem problemu, warto porównać wyniki działania wszystkich algorytmów.



\end{document}